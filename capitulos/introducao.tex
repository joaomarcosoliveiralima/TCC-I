% ----------------------------------------------------------
% Introdução 
% Capítulo sem numeração, mas presente no Sumário
% ----------------------------------------------------------

\chapter[Introdução]{Introdução}
%\addcontentsline{toc}{chapter}{Introdução}

Ao longo da história da humanidade, foram presenciadas várias formas de geração e/ou utilização de energia para diversos fins, desde a utilização da força braçal do homem e de animais, do aproveitamento das forças da água e do vento, da utilização do vapor, até a descoberta e utilização da energia elétrica. Todas estas formas de utilização de energia tiveram grande importância em suas respectivas épocas e foram de extrema importância para o desenvolvimento da humanidade, o mesmo ocorre na atualidade com a utilização da energia elétrica que se tornou uma das principais fontes de energia no mundo, a qual já proporcionou inúmeros avanços em diversas áreas como transporte, industrial, médicas, comunicação, entre outras \cite{jones1991electrical}.

Vários fatores contribuíram para o crescimento desta como fonte principal de energia, dentre os quais podemos destacar, a possibilidade de transformação desta em praticamente qualquer outra forma de energia e a relativa facilidade de seu transporte em grandes distâncias através de linhas de alta tensão. No entanto, no Brasil, o setor energético vem enfrentando uma crise de investimentos, e a escassez de chuvas nas regiões onde se encontram os maiores reservatórios de água essenciais para a geração de energia elétrica pelas usinas hidrelétricas, além do aumento da demanda por parte dos consumidores, consequentemente causa a necessidade da utilização das termoelétricas, esses fatores podem ser citados como contribuintes para o aumento gradativo no custo da produção de energia \cite[p. 11]{alissonUTFPR2016}. 

%A energia elétrica pode ser transformada de maneira direta em qualquer outra energia seja ela mecânica, química, térmica, entre outras. Além de apresentar facilidade de transporte e grande alcance através das linhas de alta tensão. Dessa forma a energia elétrica se tornou uma das fontes energéticas mais utilizadas no século XX, o que acabou provocando um crescimento vertiginoso na sua produção. No entanto no Brasil o setor energético vem enfrentando uma crise de investimentos, e a escassez de chuvas nas regiões onde se encontram os maiores reservatórios de água essenciais para a geração de energia elétrica pelas usinas hidrelétricas, além do aumento da demanda por parte dos consumidores, consequentemente causa a necessidade da utilização das termoelétricas, esses fatores podem ser citados como contribuintes para o aumento gradativo no custo da produção de energia \cite[p. 11]{alissonUTFPR2016}.

Esse aumento no custo da produção causado pela necessidade do uso da geração térmica, implica no aumento das tarifas de energia elétrica. Por mais que as usinas térmicas apresentem um custo de geração mais elevado, representam a segurança do abastecimento, e funcionam como uma fonte alternativa de suplementação do sistema quanto por motivos de escassez de chuvas as hidrelétricas não possuem condições suficientes para gerar toda a energia que o país necessita \cite{tancredi2013brasil}. Hoje é possível notar que a energia elétrica está presente em todos os setores da economia, então uma melhor gestão do consumo de energia torna-se uma tarefa interessante para fins de economia e redução de gastos.

% Em razão ao aumento no custo da produção de energia elétrica, o que consequentemente causa um aumento no preço repassado ao consumidor.

Normalmente os consumidores só possuem acesso limitado ao seu consumo de energia elétrica, isto é, os mesmos só possuem acesso a estas informações no final do mês. Além disso o usuário também não possui qualquer forma de realizar o monitoramento ou controle individual de cada um de seus equipamentos o que ofereceria uma melhor forma de controle sobre seus gastos, onde seria possível perceber os equipamentos com maior consumo energético e tomar alguma providência. Entretanto para tentar solucionar este problema o mercado disponibiliza alguns sistemas que executam o monitoramento em tempo real como o \textit{OWL CM160}\footnote{https://www.efimarket.pt/monitor-energetico-owl-cm-160} e \textit{Energy Detective Pro}\footnote{http://www.theenergydetective.com/prodocs} que foram utilizados como base neste estudo, ambos possuem conexão \textit{wireless} e atendem também o sistema trifásico. No entanto, são soluções que necessitam ser importadas além de não possuírem suporte de tradução e estão a um preço consideravelmente alto. Nosso produto a ser desenvolvido visa a redução de custos de implementação deste sistema em relação aos disponíveis no mercado para que sejam acessíveis ao consumidor, além de  permitir o monitoramento em tempo real de cada um dos seus equipamentos, proporcionar ao usuário uma conexão remota de qualquer local, notificar o consumidor quando sua demanda de energia estiver próxima de ser alcançada e fornecer uma fácil instalação.

\section{Motivação}\label{sec:motivação}
% \addcontentsline{toc}{section}{Motivação}
A motivação deste trabalho advém da necessidade de permitir que o usuário possa monitorar e controlar o consumo de seus equipamentos remotamente através de uma aplicação móvel para smartphones, a fim de conscientizá-lo e permitir que este possa entender melhor seus vícios de consumo.

\section{Objetivos}\label{sec:objetivos}
\subsection{Objetivos Gerais}\label{sec:objetivos_gerais}
Desenvolver um equipamento capaz de medir o consumo de energia elétrica de residências e estabelecimentos em geral, bem como transmitir essas informações através da internet e disponibilizá-las em uma aplicação para \textit{smartphones} para que o usuário possa acompanhar o consumo de energia elétrica de sua residência ou estabelecimento de qualquer lugar.

\subsection{Objetivos Específicos}\label{sec:objetivos_especificos}

\begin{itemize}
	\item Pesquisar os sistemas de monitoramento de consumo de energia elétrica disponíveis no mercado;
	\item Desenvolver um protótipo do produto para monitoramento de demandas energéticas, integrando a aquisição de dados e a determinação da potência, englobando \textit{hardware} e \textit{software};
    \item Desenvolver um aplicativo que permita ao usuário acompanhar e controlar o seu consumo energético;
    \item Realizar a verificação e validação do produto por meio de experimentos em campo, comparando os valores adquiridos com os obtidos por instrumentos confiáveis de medição.
\end{itemize}

\section{Organização}\label{sec:organização}

%\addcontentsline{toc}{section}{Objetivos}
